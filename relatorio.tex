%%%%%%%%%%%%%%%%%%%%%%%%%%%%%%%%%%%%%%%%%%%%%%%%%%%%%%%%%%%%%%%%%%%%%%%%%%%%%%%%%%%%%%
% Modelo de relatório de Disciplina de MLP a partir da
% classe latex iiufrgs disponivel em http://github.com/schnorr/iiufrgs
%%%%%%%%%%%%%%%%%%%%%%%%%%%%%%%%%%%%%%%%%%%%%%%%%%%%%%%%%%%%%%%%%%%%%%%%%%%%%%%%%%%%%%

%%%%%%%%%%%%%%%%%%%%%%%%%%%%%%%%%%%%%%%%%%%%%%%%%%%%%%%%%%%%%%%%%%%%%%%%%%%%%%%%%%%%%%
% Definição do tipo / classe de documento e estilo usado
%%%%%%%%%%%%%%%%%%%%%%%%%%%%%%%%%%%%%%%%%%%%%%%%%%%%%%%%%%%%%%%%%%%%%%%%%%%%%%%%%%%%%%
%
\documentclass[rel_mlp]{iiufrgs}

%%%%%%%%%%%%%%%%%%%%%%%%%%%%%%%%%%%%%%%%%%%%%%%%%%%%%%%%%%%%%%%%%%%%%%%%%%%%%%%%%%%%%%
% Importação de pacotes
%%%%%%%%%%%%%%%%%%%%%%%%%%%%%%%%%%%%%%%%%%%%%%%%%%%%%%%%%%%%%%%%%%%%%%%%%%%%%%%%%%%%%%
% (a A seguir podem ser importados os pacotes necessários para o documento, de acordo 
% com a necessidade)
%
\usepackage[brazilian]{babel}	    % para texto escrito em pt-br
\usepackage[utf8]{inputenc}         % pacote para acentuação
\usepackage{graphicx}         	    % pacote para importar figuras
\usepackage[T1]{fontenc}            % pacote para conj. de caracteres correto
\usepackage{times}                  % pacote para usar fonte Adobe Times
\usepackage{enumerate}              % para lista de itens com letras
\usepackage{breakcites}
\usepackage{titlesec}
\usepackage{enumitem}
\usepackage{titletoc}               
\usepackage{listings}			    % para listagens de código-fonte
\usepackage{mathptmx}               % p/ usar fonte Adobe Times nas formulas matematicas
\usepackage{url}                    % para formatar URLs
%\usepackage{color}				    % para imagens e outras coisas coloridas
%\usepackage{fixltx2e}              % para subscript
%\usepackage{amsmath}               % para \epsilon e matemática
%\usepackage{amsfonts}
%\usepackage{setspace}			    % para mudar espaçamento dos parágrafos
%\usepackage[table,xcdraw]{xcolor}  % para tabelas coloridas
%\usepackage{longtable}             % para tabelas compridas (mais de uma página)
%\usepackage{float}
%\usepackage{booktabs}
%\usepackage{tabularx}
%\usepackage[breaklinks]{hyperref}

\usepackage[alf,abnt-emphasize=bf]{abntex2cite}	% pacote para usar citações abnt

%%%%%%%%%%%%%%%%%%%%%%%%%%%%%%%%%%%%%%%%%%%%%%%%%%%%%%%%%%%%%%%%%%%%%%%%%%%%%%%%%%%%%%
% Macros, ajustes e definições
%%%%%%%%%%%%%%%%%%%%%%%%%%%%%%%%%%%%%%%%%%%%%%%%%%%%%%%%%%%%%%%%%%%%%%%%%%%%%%%%%%%%%%
%

% define estilo de parágrafo para citação longa direta:
\newenvironment{citacao}{
    %\singlespacing
    %\footnotesize
    \small
    \begin{list}{}{
        \setlength{\leftmargin}{4.0cm}
        \setstretch{1}
        \setlength{\topsep}{1.2cm}
        \setlength{\listparindent}{\parindent}
    }
    \item[]}{\end{list}
}

% adiciona a fonte em figuras e tabelas
\newcommand{\fonte}[1]{\\Fonte: {#1}}

% Ative o seguinte caso alguma nota de rodapé fique muito longa e quebre entre múltiplas
% páginas
%\interfootnotelinepenalty=10000

%%%%%%%%%%%%%%%%%%%%%%%%%%%%%%%%%%%%%%%%%%%%%%%%%%%%%%%%%%%%%%%%%%%%%%%%%%%%%%%%%%%%%%
% Informações gerais                                   
%%%%%%%%%%%%%%%%%%%%%%%%%%%%%%%%%%%%%%%%%%%%%%%%%%%%%%%%%%%%%%%%%%%%%%%%%%%%%%%%%%%%%%

% título
\title{Relatório Parcial do Projeto da Disciplina de MLP do Instituto de Informática da UFRGS} 

% autor
\author{Autores(s)}{Aluno(s)} % {sobrenome}{nome}
\author{Morais}{Bruno} % {sobrenome}{nome} 1 para cada aluno
\author{Pakulski}{Gabriel}
\author{da Silva}{Marcos Vinicios}       % coloquem o  nomezinho de vcs                   AQUIIIII

% Professor orientador da disciplina
\advisor[Prof.~Dr.]{Mello Schnorr}{Lucas}

% Nome do(s) curso(s):
\course{Curso de Graduação em Ciência da Computa{\c{c}}{\~a}o e Engenharia de Computação}

% local da realização do trabalho 
\location{Porto Alegre}{RS} 

% data da entrega do trabalho (mês e ano)
\date{12}{2018}


% Palavras chave
\keyword{UFRGS}
\keyword{MLP}
\keyword{Implementação}
\keyword{Jogo}
\keyword{Space Invaders}


%%%%%%%%%%%%%%%%%%%%%%%%%%%%%%%%%%%%%%%%%%%%%%%%%%%%%%%%%%%%%%%%%%%%%%%%%%%%%%%%%%%%%%
% Início do documento e elementos pré-textuais
%%%%%%%%%%%%%%%%%%%%%%%%%%%%%%%%%%%%%%%%%%%%%%%%%%%%%%%%%%%%%%%%%%%%%%%%%%%%%%%%%%%%%%

% Declara início do documento
\begin{document}

% inclui folha de rosto 
\maketitle      

\selectlanguage{brazilian}



%%%%%%%%%%%%%%%%%%%%%%%%%%%%%%%%%%%%%%%%%%%%%%%%%%%%%%%%%%%%%%%%%%%%%%%%%%%%%%%%%%%%%
% Aqui comeca o texto propriamente dito
%%%%%%%%%%%%%%%%%%%%%%%%%%%%%%%%%%%%%%%%%%%%%%%%%%%%%%%%%%%%%%%%%%%%%%%%%%%%%%%%%%%%%

%espaçamento entre parágrafos
%\setlength{\parskip}{6 pt}

\selectlanguage{brazilian}



%%%%%%%%%%%%%%%%%%%%%%%%%%%%%%%%%%%%%%%%%%%%%%%%%%%%%%%%%%%%%%%%%%%%%%%%%%%%%%%%%%%%%
% Introdução
%
\chapter{Introdução} \label{intro}

Este relatório tem por objetivo explicitar e justificar a escolha das características que os autores utilizaram para elaboração do Projeto Final da disciplina de Modelos de Linguagem de Programação - UFRGS. O Projeto trata-se de 2 implementações de um "Desafio" escolhido dentre uma lista disposta pelo Orientador da Disciplina com uma linguagem de programação proposta também pelo mesmo, sendo uma implementação utilizando paradigmas de linguagem Orientada a Objetos e a outra utilizando paradigmas de linguagem funcional.


%%%%%%%%%%%%%%%%%%%%%%%%%%%%%%%%%%%%%%%%%%%%%%%%%%%%%%%%%%%%%%%%%%%%%%%%%%%%%%%%%%%%%
% Capítulo 2
%
\chapter{O jogo - Space Invaders}


Space Invaders é um jogo de videogame de arcade lançado em 1978. Space Invaders foi um dos primeiros jogos de tiro com gráfico bidimensional. O objetivo é destruir ondas de naves com uma espaçonave humana para ganhar o maior número de pontos possível. Para construção do jogo houve a inspiração na mídia popular da época, como Guerra dos Mundos e Star Wars.

Escolheu-se este jogo pela fácil usabilidade, por ser um ótimo e renomado arcade clássico dos últimos tempos, e além disso, por conter elementos de fácil abstração entre o mundo real e implementações utilizando os paradigmas de Orientação a Objetos(OO).

%%%%%%%%%%%%%%%%%%%%%%%%%%%%%%%%%%%%%%%%%%%%%%%%%%%%%%%%%%%%%%%%%%%%%%%%%%%%%%%%%%%%%
% Capítulo 3
%
\chapter{A linguagem - Scala}

Para implementação do Projeto escolhemos a linguagem de programação Scala, pois a mesma é compatível com Java, e possui muitos aspectos semelhantes a ela, e além de seguir os paradigmas de OO possui muitas características de linguagens de programação funcional, sendo uma ótima linguagem para atender a proposta do Projeto.

Um dos outros motivos que nos levou a escolher a linguagem de programação Scala é o seguinte: a escalabilidade, como o nome sugere, pois é uma linguagem multi-paradigma em que só existem objetos e métodos. Qualquer função dentro da linguagem é uma chamada de método e quaisquer valores são classes (mesmo os "tipos primitivos" são classes, diferentemente de Java, onde \textit{float} e\textit{ Float} são elementos diferentes). Ainda assim, é possível importar classes e métodos definidos em Java, ou seja, além de todos os pacotes da linguagem Scala é possível incorporar pacotes Java, evitando-se retrabalho.

Mesmo sendo uma linguagem derivada, entre outras, de Java, programas em Scala são menos verbosos, mas conservam a qualidade da legibilidade do código. Scala é uma linguagem de programação estáticamente tipada e, com isso, o sistema de tipos nos oferece segurança através de propriedades verificáveis (e.g. não é possível fazer adição de inteiros e strings, a menos que seja criado um método responsável pelo operação) e, com isso, o algoritmo de \textit{type inference} se torna o responsável pela redução da verbosidade da linguagem.

Uma das principais aplicações que utilizam Scala é o Apache Spark, que consiste numa ferramenta para processar grandes volumes de dados de forma paralela e distribuída que utiliza a linguagem Scala na implementação de seus componentes de \textit{back-end}.
%%%%%%%%%%%%%%%%%%%%%%%%%%%%%%%%%%%%%%%%%%%%%%%%%%%%%%%%%%%%%%%%%%%%%%%%%%%%%%%%%%%
% Referências 
%%%%%%%%%%%%%%%%%%%%%%%%%%%%%%%%%%%%%%%%%%%%%%%%%%%%%%%%%%%%%%%%%%%%%%%%%%%%%%%%%%%
%

%\bibliographystyle{abnt}

\bibliographystyle{abntex2-alf}


\bibliography{biblio} % arquivo que contém as referências (no formato bib). Colocar as suas lá (se tiver dúvida sobre como adicionar novas referências, usar o software JabRef ou Medley)


\noindent \textit{ODERSKY, M.; SPOON, L.; VENNERS, B. Programming in Scala. Mountain View, California: Artima Press, 2008. p3 - 27.}

\noindent\textit{ https://pt.wikipedia.org/wiki/Space_Invaders.}\textit{ Acesso em 17/09/2018.}

\noindent\textit{ https://www.devmedia.com.br/introducao-ao-apache-spark/34178}. \textit{Acesso em 18/09/2018.}

\noindent\textit{https://blog.insightdatascience.com/a-newbies-guide-to-scala-and-why-it-s-used-for-distributed-computing-979070a9f8. Acesso em 19/09/2018.}
\end{document}
